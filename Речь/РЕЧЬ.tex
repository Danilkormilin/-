\documentclass[14pt, a4paper]{extreport}
\usepackage{susu}
\usepackage{subcaption}
\DeclareCaptionSubType*{figure}
\renewcommand\thesubfigure{\asbuk{subfigure})}
\DeclareCaptionLabelFormat{russian}{\MakeUppercase{#1~#2}}
\captionsetup[subfigure]{labelformat=russian,font=normal}
\begin{document}
	\textbf{СЛАЙД 1}
	
	Здравствуйте уважаемые председатель и члены аттестационной комиссии. Тема моей выпускной кваллификационной работы -- Решение задачи сегментации
	факела выбросов на основе данных тепло-видео системы наблюдения. На сегодняшний день серьезно встает вопрос загрязнения окружающей среды и сегментация факела выбросов поможет эфективней контролирова выбросы предприятий.
	
	\textbf{СЛАЙД 2}
	
	Целью данной работы является разработка алгоритма сегментации факела выбросов с использованием тепло-видео систем. Задачи представлены на слайде.
	
	\textbf{СЛАЙД 3}
	
	Мной были исследованы современные методы контроля вредных выбросов. Их классификация представленна на слайде. Было выявлено, что инструментальный метод является трудным в исполнении и дорогим, тогда как расчетный метод -- недостаточно точным. Одним из решений этой проблемы является сегментация факела выбросов.
	
	\textbf{СЛАЙД 4}
	
	На данном слайде вы можете увидеть основные преемущества использования тепло-видео систем наблюденя с иллюстрациями. 
	
	\textbf{СЛАЙД 5}
	
	Постановка задачи сегментации факела выбросов звучит следующимобразом: X -- пространство пар изображений и соответствующих им матрц температур. Z -- пространство масок соответствующей разменрности где каждый пиксель отражает вероятность принадлежности к факелу. Необходимо восстановить функцию (1).
	
	\textbf{СЛАЙД 6}
	
	Первым этапом является подготовка данных для сегментации. Она в свою очередь тоже делится на подзадачи.
	
	\textbf{СЛАЙД 7}
	
	На даном слайде представлен пример снимков полученных с тепловизора. 
	
	\textbf{СЛАЙД 8}
	
	На данном слайде вы можете увидеть схему алгоритма сохранения снимков, основным требованием к которой является синхроная запись оптических и тепловых снимков. 
	
	\textbf{СЛАЙД 9}
	
	Следующим этапом является преобразование теплового снимка к матрице относительных температур, или к черно-белому изображению как показано на рисунке 5. 
	
	\textbf{СЛАЙД 10}
	
	Более формально необходимо классифицировать цвета по 256 классам. Для этого было решено использовать модель классификации классического машинного обучения с учителем <<k ближайших соседей>>, Использующую структуру данных <<k-мерное дерево>> позволяющую получить ближайшую точку.
	
	\textbf{СЛАЙД 11}
	
	На рисунке 7 вы можете увидеть схему алгоритма преобразования цветов.
	
	
	\textbf{СЛАЙД 12}
	
	Для анализа точности алгоритма была сформулированна следующая метрика точности, соответсвующая среднему евклидовому расстоянию между интенсивностью пикселей, которую вы можете увидеть на слайде. 
	
	\textbf{СЛАЙД 13}
	
	Данная метрика точности была посчитана для изображений с разной степенью сжатия.
	
	\textbf{СЛАЙД 14}
	
	Полный алгоритм подготовки данных представлен на рисунке 9.
	
	\textbf{СЛАЙД 15}
	
	После подготовки данных мы приступаем к решению задачи сегментации. Данная задача делится на 2 подзадачи. Первой подзадачей является задача детекции трубы. Зачастую труба температура трубы выше чем температуры выбросов, как следствие алгоритмы сегментации основанные на температуре сегментируют также трубу. В зависимоти от отдаления от поворота камеры и времени суток вид трубы и ее положение может отличаться. Алгоритм делится 
	
	\textbf{СЛАЙД 16}
	
	 Алгоритм делится на три этапа: первый этап получение ключевых точек на изображении образце и на входном изображении. второй этап классификация ключевых точек по точкам на изображении образце. третий этап восстановление координат прямоугольника с трубой. Этапы 1 и 3 представленны на рисунке 10 в виде схем.
	 
	 \textbf{СЛАЙД 17}
	 
	 На рисунке 11 можно увидеть пример работы алгоритма детекции трубы.
	 
	 \textbf{СЛАЙД 18}
	 
	 Следующим этапом является непосредственно сегментация. Так как матрица температур является черно белым изображеним с низкой контрастностью и большим количеством шумов было решено использовать алгоритм сегментации водоразделом. Коротко данный алгоритм можно описать так градиент функции представляется в виде топологической поверхности, в локальных минимумах проделываются отверстия, и начинается затопление этой поверхности. В месте соединения воды появляется водораздел являющийся границей классов сегментации.
	 
	\textbf{СЛАЙД 19}
	 
	 Схема алгоритма сегментации методом водораздела представлена на рисунке 13.
	 
	 \textbf{СЛАЙД 20}
	 
	 Для тестирования алгоритма была сформирована тестовая выборка и на рисунке 14 можно увидеть пример работы алгоритма сегментации.
	 
	 \textbf{СЛАЙД 21}
	 
	 Данная тестовая выборка была в ручную размечена и на рисунке 15 представлеенна визуализация разницы масок, размеченых вручную и масок полученных с помощью алгоритма.
	 
	 \textbf{СЛАЙД 22}
	 
	 Для оценки точности решено было использовать коэффициент Серенсена - Дайса, вычисляемый по формуле 3. Здесь TP FP и FN -- площади соответствующих сегментов. Метрика является компромиссом между полнотой (отношение TP к общему количеству истинных объектов) и точностью (отношение TP к общему количеству предсказанных объектов). Он позволяет оценивать как качество сегментации, так и ее объем.
	 
	 \textbf{СЛАЙД 23}
	 
	 Получившаяся средняя точность - 86,2\%. На рисунке 16 представлен график точности от номера кадра.
	 
	 \textbf{СЛАЙД 24}
	 
	 В ходе работы был разработан алгоритм сегментации факела выбросов, исползующий данные тепло-видео систем наблюдения. Данная работа выполнена в рамках проекта <<Экомонитор>> кафедры <<Прикладная математика и программирование>>.
	 
	 Была разработана математическая модель алгоритмов подготовки данных, детекции трубы и сегментации факела выбросов методом водораздела. Данные алгоритмы были разработаны и реализованы.

	 Было произведено тестирование разработаного программного модуля. Проведенный анализ точности работы алгоритма позволяет говорить о целесообразности применения данного метода.. Таким образом цель работы достигнута, а все поставленные задачи
	 решены
	 
	 \textbf{СЛАЙД 25}
	 
	 Спасибо за внимание, готов ответить на ваши вопросы
	 решены
\end{document}
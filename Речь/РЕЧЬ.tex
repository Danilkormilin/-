\documentclass[14pt, a4paper]{extreport}
\usepackage{susu}
\usepackage{subcaption}
\DeclareCaptionSubType*{figure}
\renewcommand\thesubfigure{\asbuk{subfigure})}
\DeclareCaptionLabelFormat{russian}{\MakeUppercase{#1~#2}}
\captionsetup[subfigure]{labelformat=russian,font=normal}
\begin{document}
	\textbf{СЛАЙД 1}
	
	Здравствуйте уважаемые председатель и члены государственной экзаменационной комиссии. Тема моей выпускной кваллификационной работы -- Решение задачи сегментации
	факела выбросов на основе данных тепло-видео системы наблюдения. Сегодня очень важен вопрос загрязнения окружающей среды и сегментация факела выбросов поможет эфективней контролировать выбросы предприятий.
	
	\textbf{СЛАЙД 2}
	
	Целью данной работы является разработка алгоритма сегментации факела выбросов с использованием тепло-видео систем. Задачи представлены на слайде.
	
	\textbf{СЛАЙД 3}
	
	Мной были исследованы современные методы контроля вредных выбросов. Их классификация представленна на слайде. Было выявлено, что инструментальный метод является трудным в исполнении и дорогим, тогда как расчетный метод -- недостаточно точным. Одним из решений этой проблемы является сегментация факела выбросов.
	
	\textbf{СЛАЙД 4}
	
	Для сегментации выбросов можно использовать тепловизоры, так как выбросы имеют высокую температуру, что обеспесивает их лучшую видимость на тепловых снимках, в отличии от оптических. Также они дешевле чем газоанализаторы. 
	
	\textbf{СЛАЙД 5}
	
	Постановка задачи сегментации факела выбросов звучит следующим образом: X -- пространство пар изображений и соответствующих им матрц температур. Z -- пространство масок соответствующей разменрности где каждый пиксель отражает вероятность принадлежности к факелу. Необходимо восстановить функцию (1). (ПОКАЗЫВАЙ НА КАРТИНКИ!!!)
	
	\textbf{СЛАЙД 6}
	
	Задача сегментации факела выбросов делится на 3 основных этапа. Это подготовка даных (ТЫК), Детекция трубы (ТЫК), Сегментация факела (ТЫК).
	
	\textbf{СЛАЙД 7}
	
	Оссобенностью использованного тепловизора является вывод матрицы температур в формате цветного 3-х канального  сжатого изображения, поэтому требуется переход к одноканальному изображению в оттенках серого, где черный соответствует холодному, белый -- горячему.
	
	\textbf{СЛАЙД 8}
	
	Поэтому необходимо классифицировать цвета по 256 классам. Из за сжатия и искажения цветов было решено использовать модель классификации <<k ближайших соседей>>, которая для каждого цвета находит ближайший цвет из тестовой выборки и классифицирует его классом этого цвета.
	
	\textbf{СЛАЙД 9}
	
	Здесь представленна схема этого алгоритма. Ключевой момент -- подготовка классификатора цветов, для этого обучаем модель на 256 оттенках серого и классифицируем все RGB пространство. 
	
	\textbf{СЛАЙД 10}
	
	Для анализа точности алгоритма была сформулированна следующая метрика точности, соответсвующая среднему евклидовому расстоянию между интенсивностью пикселей, которую вы можете увидеть на слайде. 
	
	\textbf{СЛАЙД 11}
	
	Здесь видно как меняется точность в зависимости от степени сжатия изображения.
	
	\textbf{СЛАЙД 12}
	
	Здесь показан пример результата подготовки данных до подготовки и после, теловизионное изображение преобразовано к одноканальному формату, оба изображения имеют один размер.
	
	\textbf{СЛАЙД 13}
	
	2 этап -- задача детекции трубы. Зачастую температура трубы выше чем температуры выбросов, как следствие алгоритмы сегментации основанные на температуре сегментируют и трубу. Поэтому нам необходима детекция трубы заданной на образце
	
	 Алгоритм делится на три шага: первый шаг получение ключевых точек на изображении образце и на входном изображении. Второй шаг -- классификация ключевых точек по точкам на изображении образце. третий шаг восстановление координат прямоугольника с трубой. 
	 
	 \textbf{СЛАЙД 14}
	 
	 Шаги 1 и 3 представленны в виде схем. Получаем пирамиду изображений в разной степени размытых фильтром гаусса, после каждую точку проверяем на локальный экстремум в пирамиде и контрастность, и добавляем дескриптор в случае успеха.
	 для востановления координат мы некоторое количесво раз выбираем 2 случайные ключевые точки, восстанавливаем по ним прямоугольник и обновляем лучшый результат.
	 
	 \textbf{СЛАЙД 15}
	 
 	3 этап -- сегментация методом водораздела, он был выбран потому что матрица температур является черно белым изображеним с низкой контрастностью, а также потому что данный алгоритм позволяет работать с маркерами. Простыми словами можно описать алгоритм следующим образом: модуль градиента функции представляется в виде поверхности, в маркерах проделываются отверстия, и начинается затопление этой поверхности. В месте соединения воды появляется водораздел являющийся границей классов сегментации.
	 
	\textbf{СЛАЙД 16}
	 
	Тут можно увидеть алгоритм сегментации. Находятся маркеры, классу выбросов соответствует самая <<горячая точка трубы>>, классу не выбросов соответствует самая холодная точка изображения. Вычисляется модуль градиента, преобразовывается с помощью маркеров и делится на секции уровней. Для каждой секции уровня вычисляется новая зона влияния каждого маркера. После из полученной маски исключается прямоугольник с трубой
	 
	 \textbf{СЛАЙД 17}
	 
	 Для тестирования алгоритма была сформирована синтетическая тестовая выборка и на рисунке 12 можно увидеть пример работы алгоритма сегментации.
	 
	 \textbf{СЛАЙД 18}
	 
	 Данная тестовая выборка была в ручную размечена и на рисунке 15 представлеенна визуализация разницы масок, размеченых вручную и масок полученных с помощью алгоритма. (ТЫК!!!)
	 
	 \textbf{СЛАЙД 19}
	 
	 Для оценки точности решено было использовать коэффициент Серенсена - Дайса, вычисляемый по формуле 3. Здесь TP FP и FN -- площади соответствующих сегментов. Метрика позволяет оценивать как качество сегментации, так и ее объем. Получившаяся точность -- 86,2\%
	 
	 \textbf{СЛАЙД 20}
	 
	 В ходе работы был разработан алгоритм сегментации факела выбросов, исползующий данные тепло-видео систем наблюдения. Данная работа выполнена в рамках проекта <<Экомонитор>> института естественных и точных наук.
	 
	 Была разработана математическая модель алгоритмов подготовки данных, детекции трубы и сегментации факела выбросов методом водораздела. Данные алгоритмы были разработаны, реализованы и протестированы. Таким образом цель работы достигнута, а все поставленные задачи решены
	 
	 \textbf{СЛАЙД 21}
	 
	 Спасибо за внимание, готов ответить на ваши вопросы
\end{document}